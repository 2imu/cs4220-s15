\documentclass[12pt, leqno]{article}
\usepackage{amsfonts}
\usepackage{amsmath}
\usepackage{fancyhdr}
\usepackage{hyperref}

\newcommand{\bbR}{\mathbb{R}}
\newcommand{\bbC}{\mathbb{C}}

\newcommand{\hdr}[2]{
  \pagestyle{fancy}
  \lhead{Bindel, Spring 2015}
  \rhead{Numerical Analysis (CS 4220)}
  \fancyfoot{}
  \begin{center}
    {\large{\bf #1}} \\
    Due: #2
  \end{center}
}


\begin{document} \hdr{PS 1}{Weds, Jan 28}

\paragraph*{1: By the book}
Book section 3.6, problems 1, 4, 5

\paragraph*{2: Water, water}
The dispersion relation for shallow water waves is
\[
  \omega^2 = k \left(g + \frac{T}{\rho} k^2 \right) \tanh(k h)
\]
where
\begin{align*}
  h &= \mbox{water depth} \\
  k &= \mbox{spatial wave number ($2\pi$ / wave length)} \\
  \omega &= \mbox{frequency ($2\pi$ / period)} \\
  T &= \mbox{surface tension} \\
  \rho &= \mbox{mass density} \\
  g &= \mbox{gravitational acceleration}.
\end{align*}
For water at 25C, $T/\rho = 7.2 \times 10^{-5}$ N/m$^4$, 
and the acceleration due to gravity is $g = 9.8$ m/s$^2$.
Assuming these values, write a code using Newton's method
to find $k$ given $\omega$ and $h$, assuming $kh \ll 1$.
Your routine should take the form
\begin{lstlisting}
  function k = ps1water(omega, h)
\end{lstlisting}

\end{document}
