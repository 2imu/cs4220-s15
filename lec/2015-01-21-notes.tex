\documentclass[12pt, leqno]{article}
\usepackage{amsfonts}
\usepackage{amsmath}
\usepackage{fancyhdr}
\usepackage{hyperref}

\newcommand{\bbR}{\mathbb{R}}
\newcommand{\bbC}{\mathbb{C}}

\newcommand{\hdr}[2]{
  \pagestyle{fancy}
  \lhead{Bindel, Spring 2015}
  \rhead{Numerical Analysis (CS 4220)}
  \fancyfoot{}
  \begin{center}
    {\large{\bf #1}} \\
    Due: #2
  \end{center}
}


\begin{document}
\hdr{2015-01-21}

\section{Logistics}

\begin{itemize}
\item Welcome to CS 4220/5223 / MATH 4260!
\item Please make sure you can access CMS and Piazza
\item PS1 is posted and will be due in one week
\end{itemize}

\section{Overview}

Scientific computing involves mathematics, computation, and
applications.  We will be undertaking all three, and have students in
the class who are majoring in all three (math majors, CS majors, and
majors in various science and engineering disciplines).  I have put
prerequisites on the syllabus, and we are also doing a short
assessment at the start of class.  The quiz is not graded, but if you
have trouble with these questions, you may want to come talk to me
about whether this is the right class for you.

My intention is that you should read the book or other readings in
advance, so that we can spend as much class time as possible working
problems and getting past any points of confusion.  Bring laptops to class!
Also bring paper and a writing instrument.  You'll need both.

\section{Nonlinear equations}

Consider a cannon fired on flat ground.  We can control the angle of
$\theta$, but not the muzzle velocity $V_0$.  We want to hit a
designated target.
\begin{itemize}
\item
  What is a model we can analyze computationally?
\item
  What modeling assumptions are you making?
\item
  How do you compute the target angle?
\item
  What could go wrong with the computation?
\item
  What are the sources of error?  How sensitive are you to them?
\end{itemize}

Of course, if we had infinite ammunition, we could start shooting and
adjust the angle based on how close we are to the target.  The
simplest version of this is bisection: figure out angles that
overshoot and undershoot (a bracketing interval), then test an angle
halfway between.  This will give a new bracketing interval based on
whether we overshot or undershot.

An alternate approach is to build a simplified local model of our
model, something sufficiently simple that we can solve it.  Each time
we solve, we do another simplification.

\end{document}
